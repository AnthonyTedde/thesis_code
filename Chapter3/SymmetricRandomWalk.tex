\documentclass{article}
\usepackage{amsmath}

\usepackage{Sweave}
\begin{document}
\Sconcordance{concordance:SymmetricRandomWalk.tex:SymmetricRandomWalk.Rnw:%
1 6 1 50 0 9 1 8 0 28 1 1 13 6 1 1 5 11 1 1 19 24 1}


\section{Description}
The symetric random walk will be described in this document (Mt). it covers the theory of "Stochastic Calculus for finance" Tome 2 chapter 3 section 1.

The construction of the random walk depend on the evolution of a random variable $X_i$. The previous RV can take two value at each time, like tossing a coin. $X_i$ can take the value 1 or -1.

\begin{equation}
 \label{eq:Xi}
X_i = 
\left \{{
  \begin{array}{c} 1 \\ -1 \end{array}
  }\right .
\end{equation}
 
The Symetric Random Walk is constructed by summing up the different outcome of the random variable $X_i$ from $k$ experiments:

\begin{equation}
\label{eq:SRW}
M_k = 
\sum_{j=1}^k X_j
\end{equation}

In the following lines of code, $X_i$ is randomly difined. The variable $k$ ensure to have a sufficent number of periods to further generate the scaled random walk.
It refers to the $k$ of equation~\ref{eq:SRW}.
$p$ and $q$ are the probability measure, respectively $p$ chance to get value 1 and $q$ chance to get -1 from random variable $X_i$.

\begin{Schunk}
\begin{Soutput}
[1] 1
\end{Soutput}
\end{Schunk}





\end{document}
